\section{OOP}

	\begin{itemize}
	\item Ver tipo de objeto: \textit{type(objeto)}
	\item Ver los atributos de un objeto: \textit{dir(objeto)}. Con \textit{dir(\_\_builtins\_\_)} se ven las funciones y atributos que hay antes de importar ningún módulo.
	\item Ver los métodos aplicables a un objeto \textit{objeto.\_\_methods\_\_}
	\item Ver lo que hace un método: \textit{help(objeto.método)}
	\end{itemize}
	
	\subsection{Clases}
	\begin{itemize}
	\item Definición de clases: \textit{class} Nombre(tipo):
	\item Atributos: objeto.atributo = valor
	\item Para crear una instancia de una clase: a = Nombre()
	\item Métodos: funciones definidas dentro de una clase. Para llamarlos:
		\begin{enumerate}
		\item clase.método(objeto al que se le aplica\footnote{Al objeto al que se le aplica un método (lo que va antes del punto) se le llama parámetro. Al primer parámetro de un método se le llama \textit{self} por convenio})
		\item objeto al que se le aplica.método()
		\end{enumerate}
	\item Hay overloading y polimorfismo
	\end{itemize}

	\subsection{Herencia}
	\begin{itemize}
	\item \textit{class} Hijo(Padre)
	\item La subclase coge los métodos de la superclase
	\item Si se vuelve a escribir un método en la subclase, overload
	\item Admite herencia múltiple: Hijo(Padre1,Padre2,...)
	\item Relaciones posibles entre clases:
		\begin{itemize}
		\item HAS-A: referencia a otros objetos
		\item IS-A: herencia
		\item DEPENDS-ON: cambios en una clase provocan cambios en otras
		\end{itemize}
	\end{itemize}

	\subsection{Métodos de sobrecarga de operadores}
	\noindent Los más comunes:
	\begin{itemize}
	\item Método \_\_init\_\_, para inicializar, valores por defecto. CONSTRUCTOR ($\sim$ Java)
	\item Método \_\_del\_\_, para reclamar un objeto. DESTRUCTOR
	\item Método \_\_add\_\_, +
	\item Método \_\_or\_\_, | / OR
	\item Método \_\_str\_\_, para imprimir ($\sim$ Java) 
	\item Método \_\_call\_\_, llamada a funciones, X()	
	\item Método \_\_getattr\_\_, x.atributo  
	\item Método \_\_setattr\_\_, x.atributo = valor
	\item Método \_\_getitem\_\_, para indexar, x[key] 
	\item Método \_\_setitem\_\_, asignar valor a una posición, x[key] = valor 
	\item Método \_\_len\_\_, longitud
	\item Método \_\_cmp\_\_, comparación, == / $< >$ 
 	\item Método \_\_lt\_\_, $ < $
	\item Método \_\_eq\_\_, ==
	\item Método \_\_radd\_\_, suma por la derecha
	\item Método \_\_iadd\_\_, x+
	\item Método \_\_iter\_\_, iteraciones (bucles \textit{for}, vectorización con \textit{in}, \textit{map})    
	\end{itemize}