% Copyright (C)  2015  Ekaitz Zárraga Río <ekaitzzarraga@gmail.com>.
%     Permission is granted to copy, distribute and/or modify this document
%     under the terms of the GNU Free Documentation License, Version 1.3
%     or any later version published by the Free Software Foundation;
%     with no Invariant Sections, no Front-Cover Texts, and no Back-Cover Texts.
%     You should have received a copy of the GNU Free Documentation License
%     along with this material.
%     If not, see <http://www.gnu.org/licenses/>.

\section{Conceptos básicos}

En este apartado se explican los diferentes tipos de datos que tiene Erlang y la sintaxis básica del lenguaje.


\subsection{Tipos y variables}

\subsubsection{Números}

Erlang dispone de soporte para números de coma flotante (float) y enteros (integer) y los alternará
dependiendo de sus necesidades. No le importará qué tipos estés introduciendo.

Para trabajar con diferentes bases (hexadecimal, octal, binario etc.): \textit{BASE\#NUMERO}.
\begin{lstlisting}
1> 2#111.
7
\end{lstlisting}

\subsubsection{Variables}

La primera letra siempre tendrá que ser una mayúscula o \textit{'_'}.

En Erlang se puede asignar valor a las variables una sola vez. Tiene un sentido más matemático el hecho de
asignar un valor a una variable. Si en matemáticas \textit{X=5}, \textit{X=6} no tiene sentido porque no es
una igualdad válida. En Erlang pasa algo parecido. La primera vez se definen y a partir de ese momento el
nombre de esa variable tendrá siempre asignado el valor.

\begin{lstlisting}
1> A=34.
34
2> A.
34
3> A=90.
** exception error: no match of right hand side value 90
4> 34=90.
** exception error: no match of right hand side value 90
\end{lstlisting}

Existe una variable especial llamada \textit{'_'} que siempre se comporta como si no tuviese ningún valor
asignado, siendo útil para descartar lo que se le asigne. Nunca se podrá recuperar un valor asignado a ella.

\begin{lstlisting}
1> _=12.
12
2> _.
* 1: variable '_' is unbound
\end{lstlisting}

Para descartar el contenido de las variables, puede utilizarse la función \textit{f(Variable)} que descartará
la asignación de la \textit{Variable} o simplemente \textit{f()} que descartará todas las asignaciones.

\begin{lstlisting}
1> A=3.
3
2> A.
3
3> f().
ok
4> A.
* 1: variable 'A' is unbound
\end{lstlisting}

\subsubsection{Atoms}

Los \textit{atoms} son palabras literales. Siempre empiezan por minúsculas, por eso las variables siempre
comenzarán con una letra en mayúsculas. Los \textit{atoms} deben ser puestos entre comillas simples
\textit{(')} cuando empiezan por mayúsculas o contienen caracteres especiales.

Las palabras reservadas y las funciones son un caso concreto de \textit{atom}.

Los \textit{atoms} no se liberan durante la recolección de basura. Se guardan en una tabla, por eso no deben
cargarse de forma dinámica.

\begin{lstlisting}
1> atom.
atom
\end{lstlisting}



\subsubsection{Tuplas}

Las tuplas (\textit{tuple} en inglés) son la estructura básica para agrupar variables. Se definen entre
llaves y se separa cada elemento con comas. Aquí empieza tomar importancia la variable anónima \textit{'_'},
puesto que sirve para descartar partes de la tupla que no necesitemos:

\begin{lstlisting}
11> {A,B}={14,17}.
{14,17}
12> A.
14
13> {A,B,_}={14,17,19}.
{14,17,19}
15> {A,B}={14,17,19}.
** exception error: no match of right hand side value {14,17,19}
16> {celsius, X}={kelvin, 60}.
** exception error: no match of right hand side value {kelvin,60}
\end{lstlisting}

Las tuplas pueden agrupar cualquier tipo de elemento y no todos los elementos tienen que ser del mismo tipo.

\subsubsection{Listas}

Las listas (\textit{list} en inglés) se definen entre corchetes.

Las listas se forman por una cabeza (\textit{head}) y una cola (\textit{tail}), siendo la cabeza el primer
elemento comenzando por la izquierda y la cola la lista formada por el resto. Las funciones \textit{hd()} y
\textit{tl()} sirven para obtener la cabeza y la cola de una lista, respectivamente.

Las listas pueden contener elementos de cualquier tipo.

\begin{lstlisting}
20> [1,2,3].
[1,2,3]
21> hd([1,2,3]).
1
22> tl([1,2,3]).
[2,3]
23> tl([1,2,{hola, ekaitz}]).
[2,{hola,ekaitz}]
\end{lstlisting}

Por defecto, Erlang considera las listas como cadenas de caracteres siempre que todos los elementos de las
mismas puedan decodificarse como caracteres. Esto suele ser un engorro.
\begin{lstlisting}
42> [45,46,47,89,90].
"-./YZ"
43> [45,46,47,89,90,1].
[45,46,47,89,90,1]
\end{lstlisting}



\subsection{Operadores}

\subsubsection{Comparaciones}

Las comparaciones son las habituales de otros lenguajes de programación, pero la sintaxis no es la habitual.

\begin{table}[ht]
\centering
\begin{tabular}{|l|c|}
Igualdad             & =:=           \\
Desigualdad          & =/=           \\
Igualdad numérica    & ==            \\
Desigualdad numérica & /=            \\
Mayor o igual que    & \textgreater= \\
Menor que igual que  & =\textless    \\
Mayor que	     & \textgreater  \\
Menor que	     & \textless     \\
\end{tabular}
\end{table}

La comparación numérica (== y /=) difiere de la comparación normal (=:= y =/=) en que, en el primer caso la
comparación se hace a nivel de significado numérico, es decir: 5 y 5.0 son iguales, mientras que en el
segundo caso se considerarán diferentes por tener un tipado distinto (integer vs float).

Por otro lado, se pueden comparar objetos de diferentes tipos, pero el resultado será el siguiente:
\begin{verbatim}
number < atom < reference < fun < port < pid < tuple < list < bit string
\end{verbatim}

Nota: \textit{true} y \textit{false} son \textit{atoms} y no concuerdan con ningún valor numérico como en
otros lenguajes. Como se puede apreciar en la lista superior, siempre serán mayores que cualquier número.

\subsubsection{Operadores booleanos}

Los operadores booleanos son los habituales: \textit{and}, \textit{or}, \textit{xor}, \textit{not} (como
son \textit{atoms}, se expresan en minúsculas). Para ejecutar primero el de la izquierda y luego el de la
derecha en función del primer resultado (\textit{short-circuit evaluation}\footnote{Consultar:
\url{http://en.wikipedia.org/wiki/Short-circuit_evaluation}}) utilizar \textit{andalso} y \textit{orelse}.


\subsubsection{Operadores matemáticos}

Los operadores matemáticos más comunes son: \textit{+}, \textit{-}, \textit{*}, \textit{/}, \textit{div}
(división entera) y \textit{rem} (\textit{remainder}, el resto de la división). Pueden utilizarse paréntesis
para agrupar operaciones y cambiar su prioridad como en otros lenguajes de programación o en matemáticas.

\subsubsection{Operadores para listas}
\paragraph{Añadir y quitar elementos:} Para añadir elementos a las listas se utiliza \textit{++} y para
retirarlos \textit{-- --}. Cuidado: Son right-associative:
\begin{lstlisting}
37> [1,2,3]--[1,2]++[3].
[]
\end{lstlisting}


\paragraph{Operador cons:}
El operador \textit{cons} (de constructor) se define con la pleca ($|$). Sirve tanto para hacer
coincidir patrones o para construir listas. Su objetivo es separar la cabeza de la cola en las listas de la
siguiente manera (\textit{[HEAD$|$TAIL]}):

\begin{lstlisting}
26> [Nombre | Apellidos] = ['Ekaitz', 'Zarraga', 'Rio'].
['Ekaitz','Zarraga','Rio']
27> Nombre.
'Ekaitz'
28> Apellidos.
['Zarraga','Rio']
29> [1|[2,3,4]].
[1,2,3,4]
\end{lstlisting}

Tal y como se ha dicho antes, la cola es una lista así que, a la hora de construir listas con este operador,
lo que se añada como cola siempre deberá ser una lista. [1$|$2] no es correcto, lo correcto sería:
[1$|$[2,[]]] o [1,2$|$[]].

\paragraph{List comprehensions:}
Es una manera de construir listas de forma matemática del estilo de la siguiente, que construye la lista
[3,4,5,6]:
\begin{equation}
\{x\: \epsilon\: \mathbb{N}, x \leq  4: y = x+2\}
\end{equation}

En Erlang funciona igual, sólo cambia la sintaxis:
\begin{lstlisting}
38> [2+X || X<-[1,2,3,4]].
[3,4,5,6]
\end{lstlisting}

Pueden añadirse más condiciones:
\begin{lstlisting}
39> [2+X || X<-[1,2,3,4], X rem 2 =:= 0].
[4,6]
\end{lstlisting}

\subsubsection{Operadores binarios}

Erlang es un lenguaje con un soporte fuerte para valores en binario puesto que está diseñado para las
telecomunicaciones.

\paragraph{Separación de binarios:} Los datos binarios pueden colocarse entre $<<\:$  y $>>$ para dividirlos
en secciones legibles.

%% TODO TODO TODO

