% Copyright (C)  2015  Ekaitz Zárraga Río <ekaitzzarraga@gmail.com>.
%     Permission is granted to copy, distribute and/or modify this document
%     under the terms of the GNU Free Documentation License, Version 1.3
%     or any later version published by the Free Software Foundation;
%     with no Invariant Sections, no Front-Cover Texts, and no Back-Cover Texts.
%     You should have received a copy of the GNU Free Documentation License
%     along with this material. If not, see <http://www.gnu.org/licenses/>.

\documentclass[a4paper,10pt]{article}
% Copyright (C)  2015  Ekaitz Zárraga Río <ekaitzzarraga@gmail.com>.
%     Permission is granted to copy, distribute and/or modify this document
%     under the terms of the GNU Free Documentation License, Version 1.3
%     or any later version published by the Free Software Foundation;
%     with no Invariant Sections, no Front-Cover Texts, and no Back-Cover Texts.
%     You should have received a copy of the GNU Free Documentation License
%     along with this material.
%     If not, see <http://www.gnu.org/licenses/>.

\usepackage[utf8]{inputenc}
\usepackage[spanish,es-tabla]{babel} %Euskara est  en el paquete de franc s, instalar texlive-lang-french ;)
\usepackage{parskip}
\usepackage{framed}
\usepackage{xcolor}
\usepackage{underscore}
\usepackage{verbatim}
\usepackage{listings}
    \definecolor{dkgreen}{rgb}{0,0.6,0}
    \definecolor{gray}{rgb}{0.5,0.5,0.5}
    \definecolor{mauve}{rgb}{0.58,0,0.82}
    \lstset{
      frame=tb,
      language=erlang,
      aboveskip=4mm,
      belowskip=3mm,
      showstringspaces=false,
      breaklines=true,
      columns=flexible,
      basicstyle={\small\ttfamily},
      %numbers=left,
      %numberstyle=\tiny\color{gray},
      keywordstyle=\color{blue},
      commentstyle=\color{dkgreen},
      stringstyle=\color{mauve},
      tabsize=2,
      captionpos=b
    }
\renewcommand{\lstlistingname}{Código}
\renewcommand{\lstlistlistingname}{Índice de fragmentos de código fuente}

\setlength{\oddsidemargin}{45pt}
%\setlength{\evensidemargin}{52pt}
\setlength{\textwidth}{390pt}

\usepackage[hypertexnames=false]{hyperref}
\urlstyle{sf}

\usepackage{amsfonts}

% Title Page
\title{Erlang}
\author{Ekaitz Zárraga}


\begin{document}
\maketitle

\begin{abstract}
Principalmente basado en el lo aprendido del libro ``Learn you some Erlang for great good!''
\footnote{\url{http://learnyousomeerlang.com/}}, este documento resume los conceptos básicos del lenguaje de
programación Erlang, desde su sintaxis a su filosofía.

Las explicaciones hacen referencia, en muchos casos, a conceptos a lenguajes de programación imperativa con
el fin de simplificar las explicaciones. Sin embargo es necesario que el lector disponga unos mínimos
conocimientos de programación imperativa.
\end{abstract}

\newpage
Copyright (C)  2015  Ekaitz Zárraga Río \textless ekaitz.zarraga@gmail.com\textgreater.

Permission is granted to copy, distribute and/or modify this document under the terms of the GNU Free
Documentation License, Version 1.3 or any later version published by the Free Software Foundation; with
no Invariant Sections, no Front-Cover Texts, and no Back-Cover Texts. You should have received a copy of
the GNU Free Documentation License along with this material. If not, see \url{http://www.gnu.org/licenses/}.
\newpage

\newpage
  \pagenumbering{roman}
  \tableofcontents
\newpage
\pagenumbering{arabic}

% Copyright (C)  2015  Ekaitz Zárraga Río <ekaitzzarraga@gmail.com>.
%     Permission is granted to copy, distribute and/or modify this document
%     under the terms of the GNU Free Documentation License, Version 1.3
%     or any later version published by the Free Software Foundation;
%     with no Invariant Sections, no Front-Cover Texts, and no Back-Cover Texts.
%     You should have received a copy of the GNU Free Documentation License
%     along with this material.
%     If not, see <http://www.gnu.org/licenses/>.

\section{Introducción}

En este apartado se introduce el funcionamiento general de Erlang, su tipado, sintaxis y filosofía general.

\subsection{Funcionamiento}

Erlang dispone de una máquina virtual donde se ejecutan los programas. Es parecido a lo que hace java con los
bytecodes. Los programas hechos en Erlang tienen que ser compilados para esta máquina virtual.

Los archivos de código fuente tienen la extensión '.erl', de Erlang,  y los compilados '.beam', que proviene
de \textit{Bogdan/Björn's Erlang Abstract Machine}, el nombre de la máquina virtual. Han existido otras
máquinas virtuales pero están en desuso.

Más adelante se profundizará en la compilación.

\subsection{La Shell}

Erlang dispone de una shell. Sirve como emulador para poder ejecutar comandos pero también permite cosas como
editar código en caliente etc. Para entrar: usar el comando \textit{erl}.

\begin{lstlisting}
ekaitz@DaComputa:~$ erl
Erlang R16B03 (erts-5.10.4) [source] [64-bit] [smp:8:8] [async-threads:10] [kernel-poll:false]

Eshell V5.10.4  (abort with ^G)
1>
\end{lstlisting}

La shell está basada en Emacs, usa comandos similares [Ctrl+A] para inicio de línea, [Ctrl+E] para final de
línea, etc.


Como se aprecia arriba, [Ctrl+G] sirve para abortar. Una vez pulsado permite ejecutar unos comandos simples:


\begin{lstlisting}
1>
User switch command
 --> h
  c [nn]            - connect to job
  i [nn]            - interrupt job
  k [nn]            - kill job
  j                 - list all jobs
  s [shell]         - start local shell
  r [node [shell]]  - start remote shell
  q                 - quit erlang
  ? | h             - this message
 -->
\end{lstlisting}

Para terminar las líneas introducir '.', como cuando se pone ';' en MySQL. Las expresiones pueden separarse
por comas pero sólo se mostrará el resultado de la última.

\begin{lstlisting}
1> 2/3,3*4.
12
\end{lstlisting}

% Copyright (C)  2015  Ekaitz Zárraga Río <ekaitzzarraga@gmail.com>.
%     Permission is granted to copy, distribute and/or modify this document
%     under the terms of the GNU Free Documentation License, Version 1.3
%     or any later version published by the Free Software Foundation;
%     with no Invariant Sections, no Front-Cover Texts, and no Back-Cover Texts.
%     You should have received a copy of the GNU Free Documentation License
%     along with this material.
%     If not, see <http://www.gnu.org/licenses/>.

\section{Conceptos básicos}

En este apartado se explican los diferentes tipos de datos que tiene Erlang y la sintaxis básica del lenguaje.


\subsection{Tipos y variables}

\subsubsection{Números}

Erlang dispone de soporte para números de coma flotante (float) y enteros (integer) y los alternará
dependiendo de sus necesidades. No le importará qué tipos estés introduciendo.

Para trabajar con diferentes bases (hexadecimal, octal, binario etc.): \textit{BASE\#NUMERO}.
\begin{lstlisting}
1> 2#111.
7
\end{lstlisting}

\subsubsection{Variables}

La primera letra siempre tendrá que ser una mayúscula o \textit{'_'}.

En Erlang se puede asignar valor a las variables una sola vez. Tiene un sentido más matemático el hecho de
asignar un valor a una variable. Si en matemáticas \textit{X=5}, \textit{X=6} no tiene sentido porque no es
una igualdad válida. En Erlang pasa algo parecido. La primera vez se definen y a partir de ese momento el
nombre de esa variable tendrá siempre asignado el valor.

\begin{lstlisting}
1> A=34.
34
2> A.
34
3> A=90.
** exception error: no match of right hand side value 90
4> 34=90.
** exception error: no match of right hand side value 90
\end{lstlisting}

Existe una variable especial llamada \textit{'_'} que siempre se comporta como si no tuviese ningún valor
asignado, siendo útil para descartar lo que se le asigne. Nunca se podrá recuperar un valor asignado a ella.

\begin{lstlisting}
1> _=12.
12
2> _.
* 1: variable '_' is unbound
\end{lstlisting}

Para descartar el contenido de las variables, puede utilizarse la función \textit{f(Variable)} que descartará
la asignación de la \textit{Variable} o simplemente \textit{f()} que descartará todas las asignaciones.

\begin{lstlisting}
1> A=3.
3
2> A.
3
3> f().
ok
4> A.
* 1: variable 'A' is unbound
\end{lstlisting}

\subsubsection{Atoms}

Los \textit{atoms} son palabras literales. Siempre empiezan por minúsculas, por eso las variables siempre
comenzarán con una letra en mayúsculas. Los \textit{atoms} deben ser puestos entre comillas simples
\textit{(')} cuando empiezan por mayúsculas o contienen caracteres especiales.

Las palabras reservadas y las funciones son un caso concreto de \textit{atom}.

Los \textit{atoms} no se liberan durante la recolección de basura. Se guardan en una tabla, por eso no deben
cargarse de forma dinámica.

\begin{lstlisting}
1> atom.
atom
\end{lstlisting}



\subsubsection{Tuplas}

Las tuplas (\textit{tuple} en inglés) son la estructura básica para agrupar variables. Se definen entre
llaves y se separa cada elemento con comas. Aquí empieza tomar importancia la variable anónima \textit{'_'},
puesto que sirve para descartar partes de la tupla que no necesitemos:

\begin{lstlisting}
11> {A,B}={14,17}.
{14,17}
12> A.
14
13> {A,B,_}={14,17,19}.
{14,17,19}
15> {A,B}={14,17,19}.
** exception error: no match of right hand side value {14,17,19}
16> {celsius, X}={kelvin, 60}.
** exception error: no match of right hand side value {kelvin,60}
\end{lstlisting}

Las tuplas pueden agrupar cualquier tipo de elemento y no todos los elementos tienen que ser del mismo tipo.

\subsubsection{Listas}

Las listas (\textit{list} en inglés) se definen entre corchetes.

Las listas se forman por una cabeza (\textit{head}) y una cola (\textit{tail}), siendo la cabeza el primer
elemento comenzando por la izquierda y la cola la lista formada por el resto. Las funciones \textit{hd()} y
\textit{tl()} sirven para obtener la cabeza y la cola de una lista, respectivamente.

Las listas pueden contener elementos de cualquier tipo.

\begin{lstlisting}
20> [1,2,3].
[1,2,3]
21> hd([1,2,3]).
1
22> tl([1,2,3]).
[2,3]
23> tl([1,2,{hola, ekaitz}]).
[2,{hola,ekaitz}]
\end{lstlisting}

Por defecto, Erlang considera las listas como cadenas de caracteres siempre que todos los elementos de las
mismas puedan decodificarse como caracteres. Esto suele ser un engorro.
\begin{lstlisting}
42> [45,46,47,89,90].
"-./YZ"
43> [45,46,47,89,90,1].
[45,46,47,89,90,1]
\end{lstlisting}



\subsection{Operadores}

\subsubsection{Comparaciones}

Las comparaciones son las habituales de otros lenguajes de programación, pero la sintaxis no es la habitual.

\begin{table}[ht]
\centering
\begin{tabular}{|l|c|}
Igualdad             & =:=           \\
Desigualdad          & =/=           \\
Igualdad numérica    & ==            \\
Desigualdad numérica & /=            \\
Mayor o igual que    & \textgreater= \\
Menor que igual que  & =\textless    \\
Mayor que	     & \textgreater  \\
Menor que	     & \textless     \\
\end{tabular}
\end{table}

La comparación numérica (== y /=) difiere de la comparación normal (=:= y =/=) en que, en el primer caso la
comparación se hace a nivel de significado numérico, es decir: 5 y 5.0 son iguales, mientras que en el
segundo caso se considerarán diferentes por tener un tipado distinto (integer vs float).

Por otro lado, se pueden comparar objetos de diferentes tipos, pero el resultado será el siguiente:
\begin{verbatim}
number < atom < reference < fun < port < pid < tuple < list < bit string
\end{verbatim}

Nota: \textit{true} y \textit{false} son \textit{atoms} y no concuerdan con ningún valor numérico como en
otros lenguajes. Como se puede apreciar en la lista superior, siempre serán mayores que cualquier número.

\subsubsection{Operadores booleanos}

Los operadores booleanos son los habituales: \textit{and}, \textit{or}, \textit{xor}, \textit{not} (como
son \textit{atoms}, se expresan en minúsculas). Para ejecutar primero el de la izquierda y luego el de la
derecha en función del primer resultado (\textit{short-circuit evaluation}\footnote{Consultar:
\url{http://en.wikipedia.org/wiki/Short-circuit_evaluation}}) utilizar \textit{andalso} y \textit{orelse}.


\subsubsection{Operadores matemáticos}

Los operadores matemáticos más comunes son: \textit{+}, \textit{-}, \textit{*}, \textit{/}, \textit{div}
(división entera) y \textit{rem} (\textit{remainder}, el resto de la división). Pueden utilizarse paréntesis
para agrupar operaciones y cambiar su prioridad como en otros lenguajes de programación o en matemáticas.

\subsubsection{Operadores para listas}
\paragraph{Añadir y quitar elementos:} Para añadir elementos a las listas se utiliza \textit{++} y para
retirarlos \textit{-- --}. Cuidado: Son right-associative:
\begin{lstlisting}
37> [1,2,3]--[1,2]++[3].
[]
\end{lstlisting}


\paragraph{Operador cons:}
El operador \textit{cons} (de constructor) se define con la pleca ($|$). Sirve tanto para hacer
coincidir patrones o para construir listas. Su objetivo es separar la cabeza de la cola en las listas de la
siguiente manera (\textit{[HEAD$|$TAIL]}):

\begin{lstlisting}
26> [Nombre | Apellidos] = ['Ekaitz', 'Zarraga', 'Rio'].
['Ekaitz','Zarraga','Rio']
27> Nombre.
'Ekaitz'
28> Apellidos.
['Zarraga','Rio']
29> [1|[2,3,4]].
[1,2,3,4]
\end{lstlisting}

Tal y como se ha dicho antes, la cola es una lista así que, a la hora de construir listas con este operador,
lo que se añada como cola siempre deberá ser una lista. [1$|$2] no es correcto, lo correcto sería:
[1$|$[2,[]]] o [1,2$|$[]].

\paragraph{List comprehensions:}
Es una manera de construir listas de forma matemática del estilo de la siguiente, que construye la lista
[3,4,5,6]:
\begin{equation}
\{x\: \epsilon\: \mathbb{N}, x \leq  4: y = x+2\}
\end{equation}

En Erlang funciona igual, sólo cambia la sintaxis:
\begin{lstlisting}
38> [2+X || X<-[1,2,3,4]].
[3,4,5,6]
\end{lstlisting}

Pueden añadirse más condiciones:
\begin{lstlisting}
39> [2+X || X<-[1,2,3,4], X rem 2 =:= 0].
[4,6]
\end{lstlisting}

\subsubsection{Operadores binarios}

Erlang es un lenguaje con un soporte fuerte para valores en binario puesto que está diseñado para las
telecomunicaciones.

\paragraph{Separación de binarios:} Los datos binarios pueden colocarse entre $<<\:$  y $>>$ para dividirlos
en secciones legibles.

%% TODO TODO TODO


% Copyright (C)  2015  Ekaitz Zárraga Río <ekaitzzarraga@gmail.com>.
%     Permission is granted to copy, distribute and/or modify this document
%     under the terms of the GNU Free Documentation License, Version 1.3
%     or any later version published by the Free Software Foundation;
%     with no Invariant Sections, no Front-Cover Texts, and no Back-Cover Texts.
%     You should have received a copy of the GNU Free Documentation License
%     along with this material.
%     If not, see <http://www.gnu.org/licenses/>.

\section{Módulos}

Un módulo (\textit{module} en inglés) es un conjunto de funciones agrupadas en un único archivo, definido con
un nombre con la extensión \textit{.erl}.

Para ejecutar funciones de un módulo concreto la sintaxis es la siguiente:

\begin{verbatim}
Modulo:Función(Argumentos).
\end{verbatim}

En el siguiente ejemplo \textit{io} sería el módulo, la función \textit{format} y los argumentos el string:
\textit{``Hola Mundo!''}:
\begin{lstlisting}
77> io:format("Hola Mundo!~n").
Hola Mundo!
ok
\end{lstlisting}

En los módulos hay dos tipos de cosas: Funciones y Atributos.

\paragraph{Los atributos} son el conjunto de metadatos que describen el módulo y se definen así:
  \begin{verbatim}
  -Nombre(Valor).
  \end{verbatim}

\begin{itemize}
  \item El atributo mínimo para un módulo es el nombre:
    \begin{verbatim}
    -module(NombreDelModulo).
    \end{verbatim}

  \item Para definir qué funciones serán mostradas al exterior del módulo se utiliza el atributo
    \textit{-export()} \footnote{Aridad: En inglés \textit{arity}. En el análisis
    matemático, la aridad de un operador matemático o de una función es el número de argumentos necesarios
    para que dicho operador o función se pueda calcular.}:
    \begin{verbatim}
    -export([Función1/Aridad, Función2/Aridad...]).
    \end{verbatim}

  \item Para importar funciones de otros módulos:
    \begin{verbatim}
    -import(Modulo,[Función1/Aridad, Función2/Aridad...]).
    \end{verbatim}

  \item Para definir macros: \textit{-define(Nombre,Valor).} que se utilizan \textit{?Nombre}. Por ejemplo,
  la macro \textit{-define(sub(X,Y), X-Y).} utilizada como \textit{?sub(13,5)} tomará el valor \textit{13-5}.
\end{itemize}


\paragraph{Las funciones} se definen de la siguiente manera:
  \begin{verbatim}
  Nombre(Argumentos)-> Cuerpo.
  \end{verbatim}

\begin{itemize}
  \item Nombre: Es un \textit{atom}, así que debe comenzar en minúsculas.
  \item Cuerpo: Expresiones separadas por comas, la última expresión ejecutada será el valor devuelto.
  \item Argumentos: Separados por comas.
\end{itemize}

Todos los módulos tienen (automáticamente) una función \textit{module_info/0} que muestra los metadatos del
módulo. Para llamarla:
\begin{lstlisting}
modulo:module_info().
\end{lstlisting}


Ejemplo de módulo:
\begin{lstlisting}
-module(functions).
-export([head/1,second/1]).

head([H|_]) -> H.
second([_,X|_]) -> X.
\end{lstlisting}

\subsection{Compilación}

Tal y como se ha dicho antes el código fuente de Erlang debe ser compilado para la máquina virtual. Existen
muchas formas de hacerlo:
\begin{itemize}
  \item Desde la shell de Erlang:
    \begin{lstlisting}[gobble=4]
    1> c(modulo).
    {ok,functions}
    \end{lstlisting}
    Si el módulo no se encuentra en la carpeta actual navegar usando \textit{cd(``path'').} y \textit{ls().} o
    introducir el path hasta el archivo.

  \item Desde la shell de unix:
    \begin{lstlisting}[language=bash,gobble=4]
    ekaitz@DaComputa:~$ erlc flags modulo.erl
    \end{lstlisting}

  \item Desde la shell de Erlang o desde un módulo:
    \begin{lstlisting}[gobble=4]
    compile:file(modulo).
    \end{lstlisting}
\end{itemize}





\include{Contents-ES/EscribirModulos}

\end{document}
